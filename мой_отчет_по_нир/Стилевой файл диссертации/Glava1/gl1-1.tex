\subsection{Описание модели}

Классическая модель SIR — это детерминированная модель, которая делит популяцию на три ключевые компартмента (группы) и описывает потоки между ними с помощью системы обыкновенных дифференциальных уравнений.

\vspace{1cm}
\textbf{Компартменты модели}
    \begin{itemize}
        \item[$\textbf{S}$] (Susceptible) --- Восприимчивые к инфекции
	    \item[$\textbf{I}$] (Infected) --- Инфицированные
        \item[$\textbf{R}$] (Recovered) ---Переболевшие
	\end{itemize}

\vspace{1cm}
\textbf{Параметры модели}
    %\fontsize{8.5pt}{11pt}\selectfont
        \begin{description}
        
            \item[$\beta$] --- \textbf{коэффициент заражения}. 
            Это произведение среднего числа контактов на человека в единицу времени и вероятности заражения при контакте с заразным индивидуумом. Параметр $\beta$ управляет скоростью распространения.
            
            \item[$\gamma$] --- \textbf{коэффициент выздоровления}. 
            Это величина, обратная среднему времени заразности $d$ (т.е., $\gamma = 1/d$). Если человек болеет в среднем 7 дней, то $\gamma = 1/7$.

            \item[$R_0 =\displaystyle \frac{\beta}{\gamma}$] --- \textbf{базовое репродуктивное число}. 
            Это среднее число людей, которых заразит один инфицированный человек за всё время своей болезни в полностью восприимчивой популяции.
            
        \end{description}

\newpage
% \vspace{1cm}
\textbf{Постановка задачи}

Динамика модели описывается следующей системой уравнений \cite{first}:
    \renewcommand{\arraystretch}{2.2}
    $$ \left\{
        \begin{array}{lll}
           \displaystyle \frac{dS}{dt} &=  \displaystyle -\frac{\beta\cdot  I\cdot  S}{N}, & \\
           \displaystyle \frac{dI}{dt} &=  \displaystyle\frac{\beta\cdot I \cdot S}{N} - \gamma\cdot I, &  \\
           \displaystyle \frac{dR}{dt} &=  \displaystyle\gamma\cdot I, & 
        \end{array}
        \right.
        $$
        где первое уравнение системы описывает динамику убывания восприимчивых к инфекции, второе — динамику инфицированных, а третье — динамику выздоровевших. 


