\section{ВВЕДЕНИЕ}
% \addcontentsline{toc}{section}{Введение}  
Эпидемии -- верные спутники человеческого общества на протяжении всей его истории. Борьба же с эпидемиями всегда являлась первостепенной задачей для выживания и продолжения существования. Математики с определенного момента стали активно включаться в эту борьбу, создавая различные модели для описания распростанения инфекций. Большенство моделей представляют собой динамические системы с некоторыми параметрами. В качестве первой модели для рассмотрения была выбрана SIR-модель, так как сама по себе эта модель является основой для других, более сложных моделей. 

Фундаментальная значимость SIR-модели заключается в ее способности абстрактно и наглядно формализовать ключевые процессы любой эпидемии, описывая переход людей из состояния восприимчивости (S) в состояние зараженности (I) и далее — в состояние приобретения иммунитета или удаления (R). Эта простота позволяет понять базовые закономерности, такие как скорость распространения и порог коллективного иммунитета.

Практическая востребованность этих моделей была убедительно продемонстрирована во время пандемии COVID-19, когда они активно использовались для прогнозирования нагрузки на систему здравоохранения, оценки сроков наступления пика волн и, что наиболее важно, для планирования и анализа эффективности противоэпидемических мероприятий.

