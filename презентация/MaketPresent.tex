\documentclass[12pt, aspectratio=1610]{beamer}
\usepackage{verbatim}
\usepackage{pythonhighlight}


\usepackage{beamerSFU-VKR}

\institute[ИМиФИ СФУ]{ФГАОУ ВО <<Сибирский федеральный университет>>\\
    Институт математики  и фундаментальной информатики\\
	Кафедра высшей и прикладной математики  \\                                  
}
\direction{Направление 01.04.02 Прикладная математика и информатика\\
}
\supervisor{д.ф.-м.н., профессор, Семенова Д.В.}
\reviewer{к.ф.-м.н., доцент, Строгов С.С.}
\title[Сравнительная характеристика математических моделей распространения инфекции]{Сравнительная характеристика математических моделей распространения инфекции}
\author[Нетесов И.Ф.]{Нетесов Иван Федорович}
	


\date[\today]{Красноярск\\\the\year{}}

\renewcommand*{\thefootnote}{\arabic{footnote}}
\setcounter{footnote}{0}

\begin{document}

\begin{frame}[t, plain]
\titlepage
\end{frame}

\begin{frame}
    \frametitle{Актуальность темы исследования}
    \begin{block}{}
    {\fontsize{10pt}{}\selectfont 

        Фундаментальная значимость SIR-модели заключается в ее способности абстрактно и наглядно формализовать ключевые процессы любой эпидемии, описывая переход людей из состояния восприимчивости (S) в состояние зараженности (I) и далее — в состояние приобретения иммунитета или удаления (R). Эта простота позволяет понять базовые закономерности, такие как скорость распространения и порог коллективного иммунитета.\\
        \\
        Практическая востребованность этих моделей была убедительно продемонстрирована во время пандемии COVID-19, когда они активно использовались для прогнозирования нагрузки на систему здравоохранения, оценки сроков наступления пика волн и, что наиболее важно, для планирования и анализа эффективности противоэпидемических мероприятий.}
	\end{block}
\end{frame}

\begin{frame}
    \frametitle{Цель и задачи}
    \begin{block}{Цель}
    Исследование моделей SIR и разработка программного комплекса для моделирования распространения инфекций
    \end{block}
    \begin{block}{Задачи}
    	\begin{enumerate}
    		\item Изучить классическую SIR модель
    		\item Рассмотерть другие детерменированные модели из семейства SIR
    		\item Изучить стохастические модификации SIR модели 
    		\item Разработать программую реализацию выбранных моделей
            \item Оценить сложность полученных алгоритмов 
    	\end{enumerate}
    \end{block}
\end{frame}

\begin{frame}
	\justifying
	\frametitle{Классическая SIR модель}
    Классическая модель SIR — это детерминированная модель, которая делит популяцию на три ключевые компартмента (группы) и описывает потоки между ними с помощью системы обыкновенных дифференциальных уравнений.
    \begin{block}{Компартменты модели}
        \begin{itemize}
		\item[$\textbf{S}$] (Susceptible) - Восприимчивые к инфекции
		\item[$\textbf{I}$] (Infected) - Инфицированные
		\item[$\textbf{R}$] (Recovered) - Переболевшие
	\end{itemize}
    \end{block}
	
    
    \begin{block}{Параметры модели}
    \fontsize{8.5pt}{11pt}\selectfont
        \begin{description}
        
            \item[$\beta$] - \textbf{коэффициент заражения}. Это произведение среднего числа контактов на человека в единицу времени и вероятности заражения при контакте с заразным индивидуумом. Параметр $\beta$ управляет скоростью распространения.
            
            \item[$\gamma$] - \textbf{коэффициент выздоровления}. Это величина, обратная среднему времени заразности $d$ (т.е., $\gamma = 1/d$). Если человек болеет в среднем 7 дней, то $\gamma = 1/7$.

            \item[$R_0 = \frac{\beta}{\gamma}$] - \textbf{базовое репродуктивное число}. Это среднее число людей, которых заразит один инфицированный человек за всё время своей болезни в полностью восприимчивой популяции.
            
        \end{description}
    \end{block}
\end{frame}
\begin{comment}

\begin{frame}
	\justifying
	\frametitle{Постановка задачи}
	
        Динамика модели описывается следующей системой уравнений:

        \vspace{0.5cm}
        
        \begin{align*}
            \frac{dS}{dt} &= -\frac{\beta I S}{N} \\
            \frac{dI}{dt} &= \frac{\beta I S}{N} - \gamma I \\
            \frac{dR}{dt} &= \gamma I
        \end{align*}
\end{frame}	
\end{comment}

\begin{frame}{Frame Title}
    \frametitle{Постановка задачи}
    Динамика модели описывается следующей системой уравнений:
    \begin{block}{}
        \begin{align*}
            \frac{dS}{dt} &= -\frac{\beta I S}{N} \quad \text{\textcolor{red}{← Убывание восприимчивых}} \\
            \frac{dI}{dt} &= \frac{\beta I S}{N} - \gamma I \quad \text{\textcolor{blue}{← Болеющие}} \\
            \frac{dR}{dt} &= \gamma I \quad \text{\textcolor{green}{← Выздоровевшие }}
        \end{align*}
    \end{block}
    
    \begin{block}{Пояснения}
        \begin{itemize}
            \item \textcolor{red}{Восприимчивые} теряются только за счет заражения
            \item \textcolor{blue}{Болеющие} пополняются из $S$ и уменьшаются за счет перехода в $R$
            \item \textcolor{green}{Не восприимчивые} только увеличиваются
        \end{itemize}
    \end{block}
\end{frame}


\begin{frame}
	\justifying
	\frametitle{Входные данные программы}
	\begin{table}[]
	    \centering
        \begin{tabular}{lcrp{8cm}}
            \toprule
            \textbf{Параметр} & \textbf{Значение} & \textbf{Описание} \\
            \midrule
            \hline
            $\beta$ & 0.300 & Скорость заражения \\
            \hline
            $\gamma$ & 0.100 & Скорость выздоровления \\
            \hline
            $R_0$ & 3.000 & Среднее число заражений от одного больного \\
            \hline
            $N$ & 1.000 & Общая численность населения \\
            \hline
            Пик зараженных & 303 & Максимальное число одновременно зараженных \\
            \hline
            Время пика (дни) & 30.2 & Время достижения пика эпидемии \\
            \hline
            Всего переболело & 941 & Общее число переболевших \\
            \hline
            Финальное $S$ & 59 & Люди, которые никогда не болели \\
            \hline
            Атака (\%) & 94.1\% & Процент населения, который переболел \\
            \bottomrule
        \end{tabular}
    \end{table}
\end{frame}

\begin{comment}
\begin{frame}[fragile]
	\justifying
	\frametitle{Основные результаты}
	Учтите, что в текущей конфигурации \verb|beamer| позволяет использовать для псевдокода только окружение \verb|algorithmic|.
	\begin{algorithmic}[1]%[5]	
	\scriptsize
	\Require{массив $A[n]$, содержащий $n$ элементов}
	\Ensure{отсортированный массив $A[n]$}
	\For{$j=1$ до $n-1$ с шагом 1}
		\State $f=0$
		\For{$i=0$ до $n-1-j$ с шагом 1}
			\If{$A[i]>A[i+1]$}
				\State Обменять $A[i]$, $A[i+1]$
				\State $f=1$
			\EndIf
			\If{$F=0$}
				\State Выйти из цикла
			\EndIf
		\EndFor
	\EndFor
	\end{algorithmic}
	
\end{frame}
\end{comment}

\begin{frame}{}
    \begin{figure}
        \centering
        \includegraphics[width=1\linewidth]{newplot.png}
        \caption{Динамика численности при кокретных входных данных}
        \label{fig:placeholder}
    \end{figure}
\end{frame}

\begin{frame}{Frame Title}
    \frametitle{Результаты работы программы}
    \begin{figure}
        \centering
        \includegraphics[width=1\linewidth]{newplot (1).png}
        \caption{Динамика заражения при разных входных данных}
        \label{fig:placeholder}
    \end{figure}
    
\end{frame}
\begin{comment}
    
\begin{frame}[fragile]
	\frametitle{Основные результаты}
	В презентации допускается использование двух и более колонок, путем использования окружения \verb!columns!. Выравнивание по высоте задается аргументом \verb|[T], [c], [b]|.
	\begin{columns}[T]
		\column{0.45\textwidth}
		\begin{itemize}
			\item Результат 1
			\item Результат 2
			\item \dots
		\end{itemize}		
		\column{.45\textwidth}
		\begin{figure}
			\includegraphics[width=\textwidth]{IMLogo.png}	
			\caption{Пример рисунка}
		\end{figure}		
	\end{columns}
\end{frame}
\end{comment}



\begin{frame}
	\frametitle{Модификации модели SIR}
	\justifying
	\label{Exp1}
	\begin{block}{}
		Для более точного моделирования той или иной инфекции существуют различные модификации модели SIR
	\end{block}
	\textbf{SEIR:} E — Латентно инфицированные (Exposed). Эти люди заражены, но еще не заразны.
	
	\textbf{SIRS:} учитывается потеря иммунитета. Выздоровевшие могут снова заболеть

	\textbf{Учет демографии:} Общее число популяции меняется в зависимости от рождаемости и смертности (не обязательно от болезни). Также существуют модели, которые делят популяцию на группы (дети, взрослые, старики и тд.), у которых могут быть разные коэффициенты заражения и выздоровления. 

	\textbf{Стохастические:}Учитывают случайность процессов заражения и выздоровления. Особенно важны на начальной стадии эпидемии (когда I мало) и для оценки вероятности вспышки.	
	
	%\hyperlink{Results1}{\beamerbutton{Больше результатов}}
\end{frame}


\begin{frame}
	\frametitle{Заключение}
	\justifying
	В ходе работы были достигнуты следующие результаты:
	\begin{itemize}
		\item Изучена классическая модель SIR
		\item Рассмотрены популярные детерминированные модификации модели SIR, а также стохастическая модель
		\item Разработана программная реализация модели SIR
	\end{itemize}
\end{frame}



\begin{comment}
\begin{frame}
	\frametitle{Аппробация работы}
	\justifying
	Здесь перечисляются все конференции, на которых представлялась работа, и публикации по теме работы.
	\begin{itemize}
		\item XXIII Международная конференция имени А. Ф. Терпугова <<ИНФОРМАЦИОННЫЕ ТЕХНОЛОГИИ И МАТЕМАТИЧЕСКОЕ МОДЕЛИРОВАНИЕ>> (ИТММ -- 2024)
		\item \dots
	\end{itemize}
\end{frame}
\end{comment}


\begin{frame}
	\frametitle{Основная литература}
	\begin{thebibliography}{9}
		\setbeamertemplate{bibliography item}[article]
		\bibitem{A}
		Жумартова Б. О., Ысмагул Р. С. ПРИМЕНЕНИЕ SIR МОДЕЛИ В МОДЕЛИРОВАНИИ ЭПИДЕМИЙ // Международный журнал гуманитарных и естественных наук. 2021. №12-2.
		\bibitem{B}
		Подзолков П.Н., Захарова И.Г., Киреев И.И., Кулдарев И.В. Программная реализация обобщенного стохастического подхода к компьютерному моделированию распространения эпидемий // Программные продукты и системы.
2025. Т. 38. № 3. С. 499–512. doi: 10.15827/0236-235X.151.499-512
        \bibitem{C}
        Н.Л. Семендяева, М.В. Орлов, Тан Жуй, Ян Эньпин  АНАЛИТИЧЕСКОЕ И ЧИСЛЕННОЕ ИССЛЕДОВАНИЕ МАТЕМАТИЧЕСКОЙ МОДЕЛИ SIR. — М., Шэньчжэнь: Факультет вычислительной математики и кибернетики МГУ, 2021.
        \bibitem{D}
        В. А. Адарченко, С. А. Бабань, А. А. Брагин, К. Ф. Гребёнкин, О. В. Зацепин, А. С. Козловских, В. В. Легоньков, Е. Н. Липилина, И. А. Литвиненко, П. А. Лобода, А. А. Овечкин, Г. Н. Рыкованов, С. И. Самарин, М. С. Ураков, А. Л. Фальков, К. Е. Хатунцев // МОДЕЛИРОВАНИЕ РАЗВИТИЯ ЭПИДЕМИИ КОРОНОВИРУСА ПО ДИФФЕРЕНЦИАЛЬНОЙ И СТАТИСТИЧЕСКОЙ МОДЕЛЯМ// Российского федерального ядерного центра – Всероссийского научно-исследовательского института технической физики имени академика Е. И. Забабахина, 2020.
        
	\end{thebibliography}
\end{frame}

\begin{frame}[plain]
	\centering\Huge Благодарю за внимание!
\end{frame}




\end{document}
