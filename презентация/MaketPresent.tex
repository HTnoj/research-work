\documentclass[12pt, aspectratio=1610]{beamer}
\usepackage{verbatim}


\usepackage{beamerSFU-VKR}

\institute[ИМиФИ СФУ]{ФГАОУ ВО <<Сибирский федеральный университет>>\\
    Институт математики  и фундаментальной информатики\\
	Кафедра высшей и прикладной математики  \\                                  
}
\direction{Направление 01.04.02 Прикладная математика и информатика\\
}
\supervisor{д.ф.-м.н., профессор, Семенова Д.В.}
\reviewer{к.ф.-м.н., доцент, Строгов С.С.}
\title[Сравнительная характеристика математических моделей распространения инфекции]{Сравнительная характеристика математических моделей распространения инфекции}
\author[Нетесов И.Ф.]{Нетесов Иван Федорович}
	


\date[\today]{Красноярск\\\the\year{}}

\renewcommand*{\thefootnote}{\arabic{footnote}}
\setcounter{footnote}{0}

\begin{document}

\begin{frame}[t, plain]
\titlepage
\end{frame}

\begin{frame}
    \frametitle{Актуальность темы исследования}
    \begin{block}{}
    {\fontsize{10pt}{}\selectfont 

        Фундаментальная значимость SIR-модели заключается в ее способности абстрактно и наглядно формализовать ключевые процессы любой эпидемии, описывая переход людей из состояния восприимчивости (S) в состояние зараженности (I) и далее — в состояние приобретения иммунитета или удаления (R). Эта простота позволяет понять базовые закономерности, такие как скорость распространения и порог коллективного иммунитета.\\
        \\
        Практическая востребованность этих моделей была убедительно продемонстрирована во время пандемии COVID-19, когда они активно использовались для прогнозирования нагрузки на систему здравоохранения, оценки сроков наступления пика волн и, что наиболее важно, для планирования и анализа эффективности противоэпидемических мероприятий.
    }
	\end{block}
\end{frame}

\begin{frame}
    \frametitle{Цель и задачи}
    \begin{block}{Цель}
    Исследование моделей SIR и разработка программного комплекса для моделирования распространения инфекций
    \end{block}
    \begin{block}{Задачи}
    	\begin{enumerate}
    		\item Изучить классическую SIR модель
    		\item Рассмотерть другие детерменированные модели из семейства SIR
    		\item Изучить стохастические модификации SIR модели 
    		\item Разработать программую реализацию выбранных моделей
            \item Оценить сложность полученных алгоритмов 
    	\end{enumerate}
    \end{block}
\end{frame}

\begin{frame}
	\justifying
	\frametitle{Классическая SIR модель}
    Классическая модель SIR — это детерминированная модель, которая делит популяцию на три ключевые компартмента (группы) и описывает потоки между ними с помощью системы обыкновенных дифференциальных уравнений.
    \begin{block}{Компартменты модели}
        \begin{itemize}
		\item[$\textbf{S}$] (Susceptible) - Восприимчивые к инфекции
		\item[$\textbf{I}$] (Infected) - Инфицированные
		\item[$\textbf{R}$] (Recovered) - Переболевшие (или умершие)
	\end{itemize}
    \end{block}
	
    
    \begin{block}{Параметры модели}
        \begin{description}
            \item[$\beta$] — \textbf{коэффициент заражения}. Это произведение среднего числа контактов на человека в единицу времени и вероятности заражения при контакте с заразным индивидуумом. Параметр $\beta$ управляет скоростью распространения.
            
            \item[$\gamma$] — \textbf{коэффициент выздоровления}. Это величина, обратная среднему времени заразности $d$ (т.е., $\gamma = 1/d$). Если человек болеет в среднем 7 дней, то $\gamma = 1/7$.
        \end{description}
    \end{block}
\end{frame}
\begin{comment}

\begin{frame}
	\justifying
	\frametitle{Постановка задачи}
	
        Динамика модели описывается следующей системой уравнений:

        \vspace{0.5cm}
        
        \begin{align*}
            \frac{dS}{dt} &= -\frac{\beta I S}{N} \\
            \frac{dI}{dt} &= \frac{\beta I S}{N} - \gamma I \\
            \frac{dR}{dt} &= \gamma I
        \end{align*}
\end{frame}	
\end{comment}

\begin{frame}{Frame Title}
    \frametitle{Постановка задачи}
    Динамика модели описывается следующей системой уравнений:
    \begin{block}{}
        \begin{align*}
            \frac{dS}{dt} &= -\frac{\beta I S}{N} \quad \text{\textcolor{red}{← Убывание восприимчивых}} \\
            \frac{dI}{dt} &= \frac{\beta I S}{N} - \gamma I \quad \text{\textcolor{blue}{← Болеющие}} \\
            \frac{dR}{dt} &= \gamma I \quad \text{\textcolor{green}{← Выздоровевшие или умершие }}
        \end{align*}
    \end{block}
    
    \begin{block}{Пояснения}
        \begin{itemize}
            \item \textcolor{red}{Восприимчивые} теряются только за счет заражения
            \item \textcolor{blue}{Болеющие} пополняются из $S$ и уменьшаются за счет перехода в $R$
            \item \textcolor{green}{Не восприимчивые} только увеличиваются
        \end{itemize}
    \end{block}
\end{frame}


\begin{frame}
	\justifying
	\frametitle{Основные результаты}
	Здесь и далее приводятся основные результаты работы: теоремы, доказательства, алгоритмы и пр.
	\begin{block}{Теорема 1}
		Формулировка всякой теоремы содержит три части:
		\begin{enumerate}
			\item Разъяснительная часть — множество, на котором рассматривается теорема;
			\item Условие — посылка в теореме, сформулированной в условной форме: то есть те положения, при которых заключение имеет место;
			\item Требование (или заключение) — что собственно необходимо доказать, или что об объекте утверждается.
		\end{enumerate}		
	\end{block}
\end{frame}


\begin{frame}[fragile]
	\justifying
	\frametitle{Основные результаты}
	Учтите, что в текущей конфигурации \verb|beamer| позволяет использовать для псевдокода только окружение \verb|algorithmic|.
	\begin{algorithmic}[1]%[5]	
	\scriptsize
	\Require{массив $A[n]$, содержащий $n$ элементов}
	\Ensure{отсортированный массив $A[n]$}
	\For{$j=1$ до $n-1$ с шагом 1}
		\State $f=0$
		\For{$i=0$ до $n-1-j$ с шагом 1}
			\If{$A[i]>A[i+1]$}
				\State Обменять $A[i]$, $A[i+1]$
				\State $f=1$
			\EndIf
			\If{$F=0$}
				\State Выйти из цикла
			\EndIf
		\EndFor
	\EndFor
	\end{algorithmic}
	
\end{frame}

\begin{frame}[fragile]
	\frametitle{Основные результаты}
	В презентации допускается использование двух и более колонок, путем использования окружения \verb!columns!. Выравнивание по высоте задается аргументом \verb|[T], [c], [b]|.
	\begin{columns}[T]
		\column{0.45\textwidth}
		\begin{itemize}
			\item Результат 1
			\item Результат 2
			\item \dots
		\end{itemize}		
		\column{.45\textwidth}
		\begin{figure}
			\includegraphics[width=\textwidth]{IMLogo.png}	
			\caption{Пример рисунка}
		\end{figure}		
	\end{columns}
\end{frame}


\begin{frame}
	\frametitle{Вычислительные эксперименты}
	\justifying
	\label{Exp1}
	\begin{block}{}
		Вычислительные эксперименты должны содержать цель, входные данные, описание и результат.
	\end{block}
	\textbf{Цель:} сравнить алгоритм Alg1 для решения задачи Problem с алгоритмом Alg2.
	
	\textbf{Входные данные:} перечень входных данных.

	\textbf{Описание.} Сравнение алгоритмом осуществлялось по \dots

	\textbf{Результаты.} Приводятся графики, таблицы и иные способы обнародования результатов проведённых экспериментов.	
	
	\hyperlink{Results1}{\beamerbutton{Больше результатов}}
\end{frame}

\begin{frame}
	\frametitle{Заключение}
	\justifying
	Здесь кратко излагаются основные результаты работы, достижение поставленных целей и задач, дополнительные выводы.
	\begin{itemize}
		\item Вывод 1
		\item Вывод 2
		\item Задача 1 решена, получено \dots
		\item \dots
	\end{itemize}
\end{frame}

\begin{frame}
	\frametitle{Аппробация работы}
	\justifying
	Здесь перечисляются все конференции, на которых представлялась работа, и публикации по теме работы.
	\begin{itemize}
		\item XXIII Международная конференция имени А. Ф. Терпугова <<ИНФОРМАЦИОННЫЕ ТЕХНОЛОГИИ И МАТЕМАТИЧЕСКОЕ МОДЕЛИРОВАНИЕ>> (ИТММ -- 2024)
		\item \dots
	\end{itemize}
\end{frame}

\begin{frame}
	\frametitle{Основная литература}
	Здесь приводится самая главная литература по исследованию, на которую ссылались.
	\begin{thebibliography}{9}
		\setbeamertemplate{bibliography item}[article]
		\bibitem{A}
		Ученов, А.Б. Самое важное открытие 2024~// А.Б. Ученов~/ Естетственно научный журнал. -- \No 1(2). -- 2024. -- С.~123-145.
		\bibitem{B}
		Мастеров, В.Г. О новой задаче тысячелетия~// В.Г. Мастеров~/ Математичекий вестник им. Тьюринга. -- \No 4(20). -- 2024. -- С.~1337-1350.
	\end{thebibliography}
\end{frame}

\begin{frame}[plain]
	\centering\Huge Благодарю за внимание!
\end{frame}

\begin{frame}[plain,noframenumbering,fragile]
	\frametitle{Вспомогательные слайды}	
	\justifying
	\label{Results1}
	Вспомогательные слайды следует помещать в конце презентации, после заключительного слова. Их не следует включать в общую номерацию слайдов, что достигается аргументом \verb|noframenumbering|.
	
	Для простоты навигации, можно создавать гиперссылки на слайды командами\linebreak \verb|\hyperlink| и \verb|\label|, например
	
	\hyperlink{Exp1}{\beamerbutton{Вернуться}}

\end{frame}
%%%%%%%%%%%%%%%%%%%%%%%%%%%%%%%%%%%%%%

%%%%%%%%%%%%%%%%%%%
\end{document}
