% \section{Вспомогательные команды}

% Если в работе требуется список сокращений, то по тексту, в месте первой встречи сокращения, используйте команду \verb|\index{}|, например ALT (A* with Landmark and Triangle inequality) \index{ALT -- A* with Landmark and Triangle inequality} \begin{spverbatim}\index{ALT -- A* with Landmark and Triangle inequality}\end{spverbatim}. 

% Для отображения списка сокращений в основном файле main.tex используйте команду \verb|\printindex|.


\subsection{Вычислительный эксперимент}

SIR-модель была реализована программно на языке Python для более наглядного представления зависимости динамики инфекции от входных параметров. Решение дифференциальных уравнений, представленных выше, было реализовано численно с использованием метода Рунге-Кутта-Фельберга (RK45), сходимость которого доказана[указать ссылку на работу]. В таблице ниже представлены входные данные программы.

\begin{table}[htbp]
    \centering
    \footnotesize % Еще больше уменьшаем шрифт
    \setlength{\tabcolsep}{4pt} % Уменьшаем отступы между столбцами
    \renewcommand{\arraystretch}{1.5} % Увеличиваем межстрочный интервал
    \caption{Входные данные программы}
    \begin{tabular}{lcrp}
        \hline
        \textbf{Параметр} & \textbf{Значение} & \textbf{Описание} \\
        \hline
        $\beta$ & $0{,}300$ & Скорость заражения \\
        \hline
        $\gamma$ & $0{,}100$ & Скорость выздоровления \\
        \hline
        $R_0$ & $3{,}000$ & Среднее число заражений от одного больного \\
        \hline
        $N$ & $1{,}000$ & Общая численность населения \\
        \hline
        Пик зараженных & $303$ & Максимальное число одновременно зараженных \\
        \hline
        Время пика (дни) & $30{,}2$ & Время достижения пика эпидемии \\
        \hline
        Всего переболело & $941$ & Общее число переболевших \\
        \hline
        Финальное $S$ & $59$ & Люди, которые никогда не болели \\
        \hline
        Атака ($\%$) & $94{,}1\%$ & Процент населения, который переболел \\
        \hline
    \end{tabular}
\end{table}
